%! TeX root: main.sheet.tex
% above is only for VimTex to avoid this popup

% Copyright (c) 2024 Lukas Heindl
%
% Permission is hereby granted, free of charge, to any person obtaining a copy
% of this software and associated documentation files (the "Software"), to deal
% in the Software without restriction, including without limitation the rights
% to use, copy, modify, merge, publish, distribute, sublicense, and/or sell
% copies of the Software, and to permit persons to whom the Software is
% furnished to do so, subject to the following conditions:
%
% The above copyright notice and this permission notice shall be included in all
% copies or substantial portions of the Software.
%
% THE SOFTWARE IS PROVIDED "AS IS", WITHOUT WARRANTY OF ANY KIND, EXPRESS OR
% IMPLIED, INCLUDING BUT NOT LIMITED TO THE WARRANTIES OF MERCHANTABILITY,
% FITNESS FOR A PARTICULAR PURPOSE AND NONINFRINGEMENT. IN NO EVENT SHALL THE
% AUTHORS OR COPYRIGHT HOLDERS BE LIABLE FOR ANY CLAIM, DAMAGES OR OTHER
% LIABILITY, WHETHER IN AN ACTION OF CONTRACT, TORT OR OTHERWISE, ARISING FROM,
% OUT OF OR IN CONNECTION WITH THE SOFTWARE OR THE USE OR OTHER DEALINGS IN THE
% SOFTWARE.


% ----- Math -----
\usepackage{amsmath, amssymb} % mathematical Symbols
\usepackage{nicematrix}
\NiceMatrixOptions{
	cell-space-top-limit=1pt,
	cell-space-bottom-limit=1pt,
	nullify-dots,
}

% ----- Font -----
\usepackage{fontspec} % LuaLaTeX

\let\mathbbalt\mathbb
\usepackage[math-style=ISO]{unicode-math}
\let\mathbb\mathbbalt% UNIVERSAL RESET TO ORIGINAL \mathbb (unicode-math behaviour still with \symbb)

\setmathfont{Latin Modern Math}
\setmathfont{Latin Modern Math}[version=bold,FakeBold={3}]

% ----- Packages -----
\usepackage{xspace}
\NewDocumentCommand{\zB}{}{z.\,B.\xspace}
\NewDocumentCommand{\eg}{}{e.\,g.\xspace}
\NewDocumentCommand{\tos}{}{\to\xspace}
\usepackage{scrlayer-scrpage}                 % header and footer

\usepackage{xcolor}
\usepackage[style=Latte]{catppuccinPalette}

\usepackage[english]{babel}                   % translate fixed strings and load german hyphenation
% linkcolor=CatBlue
% citecolor=CatGreen
% filecolor=CatSky
% menucolor=CatBlue
% runcolor=CatSky
% urlcolor=CatPink
%
% citebordercolor=CatGreen
% filebordercolor=CatSky
% linkbordercolor=CatBlue
% menubordercolor=CatBlue
% urlbordercolor=CatPink
% runbordercolor=CatSky
\usepackage[colorlinks=false, breaklinks=true, linkcolor=CatBlue, linkbordercolor=CatBlue, pdfborder={0 0 0.5}]{hyperref}       % make e.g. toc clickable
\usepackage[german=quotes]{csquotes}      % pick always the right quotation marks

\usepackage{todo}               % to enter todos in the document

\usepackage{graphicx}         % include graphics
\usepackage{tabularx}         % better configuration of tables
\newcolumntype{Y}{>{\small\centering\arraybackslash}X}
\usepackage{ltxtable}         % use tabularx as longtable (\LTXtable{width}{pathToTable}) where the table referenced to contains of \begin{longtable}{preamble}content\end{longtable}
\usepackage{booktabs}         % more nice rules (headrule/midrule/crule/bottomrule)

\NewDocumentCommand{\tc}{mm}{\textcolor{#1}{#2}}
\NewDocumentCommand{\co}{m}{\color{#1}}
% define own colors. Default: red,green,blue,cyan,magenta,yellow,black,gray,white,darkgray,lightgray,brown,lime,olive,orange,pink,purple,teal,violet
% mix by using color1!percentage!color2 => take <percentage> from color1 and 100-<percentage> from color2
% use [wave] to get the color at the specified wavelength (in nm)
% define via \definecolor{name}{model/rgb}{spec-list}

\usepackage{pifont} % special characters (insert via \ding{num})
\NewDocumentCommand{\error}{}{\ding{53}\xspace} % cross

%%%%%%%%%%%%%%%%%%%%%%%%%%%%
% SETUP FOR (PSEUDO-) CODE %
%%%%%%%%%%%%%%%%%%%%%%%%%%%%
% \usepackage[linesnumbered,vline,ruled,titlenotnumbered]{algorithm2e}  % for pseudocode
% \usepackage{listings}                                                 % for code
% \definecolor{myorange}{rgb}{1.0,0.4,0}
% \definecolor{mygray}{rgb}{0.4,0.4,0.4}
% \definecolor{keywordGreen}{RGB}{0,128,0}
% \definecolor{lightgrey}{rgb}{0.9,0.9,0.9}
% \lstset{
% 	rulecolor=\color{black},
% 	emphstyle=\color{blue},
% 	basicstyle=\scriptsize\sffamily,
% 	stringstyle=\color{myorange},
% 	commentstyle=\color{mygray},
% 	keywordstyle=\color{keywordGreen},
% 	keywordstyle=[2]\color{blue!70!black},
% %
% 	backgroundcolor=\color{lightgrey},
% 	frame=none,
% %
% 	numbers=left,
% 	numbersep=10pt,
% 	numberstyle=\tiny,
% %
% 	xleftmargin=2em,
% 	framexleftmargin=2em,
% %
% 	columns=fullflexible, % columns of chars
% 	tabsize=4,
% %
% 	captionpos=b,
% %
% 	gobble=4,             % ignore the first 4 characters
% 	breakautoindent=true, %
% 	postbreak=\mbox{\textcolor{red}{$\hookrightarrow$}\space},
% 	breaklines=true,
% %
% 	showstringspaces=false,
% 	showtabs=false,
% 	showspaces=false,
% 	keepspaces=true,
% %
% 	inputencoding=utf8, extendedchars=true,
% 	literate=
% 	{á}{{\'a}}1 {é}{{\'e}}1 {í}{{\'i}}1 {ó}{{\'o}}1 {ú}{{\'u}}1
% {Á}{{\'A}}1 {É}{{\'E}}1 {Í}{{\'I}}1 {Ó}{{\'O}}1 {Ú}{{\'U}}1
% {à}{{`a}}1 {è}{{`e}}1 {ì}{{`i}}1 {ò}{{`o}}1 {ù}{{`u}}1
% {À}{{`A}}1 {È}{{\'E}}1 {Ì}{{`I}}1 {Ò}{{`O}}1 {Ù}{{`U}}1
% {ä}{{\"a}}1 {ë}{{\"e}}1 {ï}{{\"i}}1 {ö}{{\"o}}1 {ü}{{\"u}}1
% {Ä}{{\"A}}1 {Ë}{{\"E}}1 {Ï}{{\"I}}1 {Ö}{{\"O}}1 {Ü}{{\"U}}1
% {â}{{\^a}}1 {ê}{{\^e}}1 {î}{{\^i}}1 {ô}{{\^o}}1 {û}{{\^u}}1
% {Â}{{\^A}}1 {Ê}{{\^E}}1 {Î}{{\^I}}1 {Ô}{{\^O}}1 {Û}{{\^U}}1
% {Ã}{{\~A}}1 {ã}{{\~a}}1 {Õ}{{\~O}}1 {õ}{{\~o}}1
% {œ}{{\oe}}1 {Œ}{{\OE}}1 {æ}{{\ae}}1 {Æ}{{\AE}}1 {ß}{{\ss}}1
% {ű}{{\H{u}}}1 {Ű}{{\H{U}}}1 {ő}{{\H{o}}}1 {Ő}{{\H{O}}}1
% {ç}{{\c c}}1 {Ç}{{\c C}}1 {ø}{{\o}}1 {å}{{\r a}}1 {Å}{{\r A}}1
% {€}{{\euro}}1 {£}{{\pounds}}1 {«}{{\guillemotleft}}1
% {»}{{\guillemotright}}1 {ñ}{{\~n}}1 {Ñ}{{\~N}}1 {¿}{{?`}}1
% }

\usepackage{multicol} % use \begin{multicols}{num} text \end{multicols} and the text will be automatically distributed
% \usepackage{multirow}

% ----- Math shortcuts -----
% short matrix
\NewDocumentCommand{\mat}{O{}m}{\ensuremath{\begin{bNiceMatrix}[c,#1]#2\end{bNiceMatrix}}}
\NewDocumentCommand{\ve}{m}{\symbf{#1}}
\NewDocumentCommand{\ma}{m}{\symbf{#1}}

% sets
\NewDocumentCommand{\set}{m}{\ensuremath{\left\{#1\right\}}}    % e.g. \set{a,b,c} for {a,b,c}
\NewDocumentCommand{\suchthat}{}{\ensuremath{\, \middle| \,}}   % e.g. \set{x \suchthat P(x)} for {x|P(x)}

% number sets
\NewDocumentCommand{\naturals}{}{\mathbb{N}}
\NewDocumentCommand{\wholes}{}{\mathbb{N}_0}
\NewDocumentCommand{\integers}{}{\mathbb{Z}}
\NewDocumentCommand{\rationals}{}{\mathbb{Q}}
\NewDocumentCommand{\reals}{}{\mathbb{R}}
\NewDocumentCommand{\complexes}{}{\mathbb{C}}
\NewDocumentCommand{\primes}{}{\mathbb{P}}

% absolute value
\NewDocumentCommand{\abs}{m}{\left|#1\right|}

% brackets with automatic sizing
\NewDocumentCommand{\rb}{m}{\ensuremath{\left(#1\right)}}                    % round brackets
\NewDocumentCommand{\Sb}{m}{\ensuremath{\left[#1\right]}}                  % square brackets
\NewDocumentCommand{\cb}{m}{\ensuremath{\left\{#1\right\}}}                  % curly brackets
\NewDocumentCommand{\ab}{m}{\ensuremath{\left\langle#1\right\rangle}}        % angled brackets

\NewDocumentCommand{\gdw}{}{\ensuremath{\Leftrightarrow}}
\NewDocumentCommand{\where}{}{\ensuremath{\;|\;}}

\DeclareMathOperator{\op}{op}
\DeclareMathOperator{\cond}{cond}
\DeclareMathOperator{\rd}{rd}

\usepackage{pgfkeys}                   % nice key management
% ----- Tikz Stuff -----
\usepackage{tikz}
\usetikzlibrary{calc,positioning,fit,intersections,shapes}  % calc for $...$ calculations, positioning for .west .south positioning
% \usetikzlibrary{graphs,graphdrawing}
% \usegdlibrary{trees}
% \tikzset{
% 	cross/.style={
% 		cross out,
% 		draw,
% 		minimum size=2*(#1-\pgflinewidth),
% 		inner sep=0pt,
% 		outer sep=0pt,
% 	}
% }

% \usepackage{pgfplots}                  % draw diagrams
% \pgfplotsset{compat=1.18}

% \pgfplotsset{
% 	every axis/.style={
% 		font=\scriptsize,
% 		line width=0.4pt,
% 		width = \textwidth,
% 		samples=25,
% 		grid=major,
% 		grid style={gray!30},
% 		axis lines = middle,
% 		axis line style={->},
% 		xticklabels=\empty,
% 		yticklabels=\empty,
% 		xlabel=u,
% 		ylabel=i,
% 		x label style={at={(axis description cs:+0.5,-0.01)},anchor=north},
% 		y label style={at={(axis description cs:-0.01,+0.5)},rotate=90,anchor=south},
% 	}
% }

% \makeatletter
% \newcommand*\bigcdot{\mathpalette\bigcdot@{.5}}
% \newcommand*\bigcdot@[2]{\mathbin{\vcenter{\hbox{\scalebox{#2}{$\m@th#1\bullet$}}}}}
% \makeatother

% config for packat priting with lua
\newlength{\packetcolsep}
\setlength{\packetcolsep}{1pt}

%%%%%%%%%%%%%%%%%%%
% CONDENSED LISTS %
%%%%%%%%%%%%%%%%%%%
\newcounter{listi}
\NewDocumentEnvironment{algo}{}{ % enumerate pendant
	\begin{list}{%
			\thelisti.%
		}{%
		\usecounter{listi}%
		\setlength{\parskip}{0ex}%
		\setlength{\topsep}{0ex}%
		\setlength{\partopsep}{0ex}%
		\setlength{\labelsep}{.25em}%
		\setlength{\parsep}{0ex}%
		\setlength{\rightmargin}{0pt}%
		\setlength{\itemsep}{0ex}%
		\setlength{\listparindent}{0pt}%
		\setlength{\labelwidth}{3em}%
		\setlength{\leftmargin}{\labelwidth}%
		\linespread{0.8}\selectfont%
	}%
}{%
	\end{list}%
}
\NewDocumentEnvironment{itemm}{}{ % itemize pendant
	\begin{list}{-}{%
		\setlength{\parskip}{0ex}%
		\setlength{\topsep}{0ex}%
		\setlength{\partopsep}{0ex}%
		\setlength{\labelsep}{.25em}%
		\setlength{\parsep}{0ex}%
		\setlength{\rightmargin}{0pt}%
		\setlength{\itemsep}{0ex}%
		\setlength{\listparindent}{0pt}%
		\setlength{\labelwidth}{3em}%
		\setlength{\leftmargin}{\labelwidth}%
	}%
}{%
	\end{list}%
}

\NewDocumentEnvironment{item2c}{}{ % itemize 2 column
	\setlength{\columnseprule}{0pt}
	\setlength\multicolsep{0pt}
	\begin{multicols}{2}
	\begin{itemm}
}{
	\end{itemm}
	\end{multicols}
}

\NewDocumentEnvironment{algo2c}{}{ % enumerate 2 column
	\setlength{\columnseprule}{0pt}
	\setlength\multicolsep{0pt}
	\begin{multicols}{2}
	\begin{algo}
}{
	\end{algo}
	\end{multicols}
}

\newcounter{posNegi}
\NewDocumentEnvironment{posNeg}{}{ % pos/neg List
	\NewDocumentCommand{\PNpos}{}{\item[\tc{CatGreen}{\texttt{+}}]}%
	\NewDocumentCommand{\PNneg}{}{\item[\tc{CatRed}{\texttt{-}}]}%
	\begin{list}{%
			\theposNegi.%
		}{%
		\usecounter{posNegi}%
		\setlength{\parskip}{0ex}%
		\setlength{\topsep}{0ex}%
		\setlength{\partopsep}{0ex}%
		\setlength{\labelsep}{.25em}%
		\setlength{\parsep}{0ex}%
		\setlength{\rightmargin}{0pt}%
		\setlength{\itemsep}{0ex}%
		\setlength{\listparindent}{0pt}%
		\setlength{\labelwidth}{3em}%
		\setlength{\leftmargin}{\labelwidth}%
	}%
}{%
	\end{list}%
}

%%%%%%%%%%%%%%%%%%%%%%%%%%%%%%%%%%%%%%%%%%%%%%%%%%%%%
% MORE LITERAL INCLUDEGRAPHICS COMMAND FOR CROPPING %
%%%%%%%%%%%%%%%%%%%%%%%%%%%%%%%%%%%%%%%%%%%%%%%%%%%%%
\pgfkeys{
	/graphicx/.cd,
	left/.initial=0cm, left/.default=0cm,
	right/.initial=0cm, right/.default=0cm,
	top/.initial=0cm, top/.default=0cm,
	bottom/.initial=0cm, bottom/.default=0cm,
	pages/.initial=-, pages/.default=-,
	width/.initial=\linewidth, width/.default=\linewidth,
	code/.initial={}, code/.default={},
}
\NewDocumentCommand{\slides}{O{}m}{
	\pgfkeys{/graphicx/.cd,left,right,top,bottom,pages,code,#1}
	\includegraphics[
		page=\pgfkeysvalueof{/graphicx/pages},
		width=\pgfkeysvalueof{/graphicx/width},
		clip,
		trim=\pgfkeysvalueof{/graphicx/left} \pgfkeysvalueof{/graphicx/bottom} \pgfkeysvalueof{/graphicx/right} \pgfkeysvalueof{/graphicx/top},
		% \pgfkeysvalueof{/graphicx/code}
		]{#2}
}

\NewDocumentCommand{\good}{}{\texttt{\textcolor{CatGreen}{+}}\xspace}
\NewDocumentCommand{\bad}{}{\texttt{\textcolor{CatRed}{-}}\xspace}

\NewDocumentCommand{\partAbl}{mm}{\frac{\partial #1}{\partial #2}}

% integrals
\NewDocumentCommand{\diff}{}{\,\mathrm{d}}
\NewDocumentCommand{\integral}{O{}O{}mm}{\int_{#1}^{#2} #3 \diff #4}

% there is also \bar
\NewDocumentCommand{\ov}{m}{\overline{#1}}

\DeclareMathOperator*{\argmax}{argmax}
\DeclareMathOperator*{\argmin}{argmin}
\DeclareMathOperator{\Var}{Var}
\DeclareMathOperator{\Expected}{E}
\DeclareMathOperator{\sign}{sign}
\DeclareMathOperator{\Cov}{Cov}
\DeclareMathOperator{\tr}{tr}
\DeclareMathOperator{\st}{s.t.}
